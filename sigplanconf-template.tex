%-----------------------------------------------------------------------------
%
%               Template for sigplanconf LaTeX Class
%
% Name:         sigplanconf-template.tex
%
% Purpose:      A template for sigplanconf.cls, which is a LaTeX 2e class
%               file for SIGPLAN conference proceedings.
%
% Guide:        Refer to "Author's Guide to the ACM SIGPLAN Class,"
%               sigplanconf-guide.pdf
%
% Author:       Paul C. Anagnostopoulos
%               Windfall Software
%               978 371-2316
%               paul@windfall.com
%
% Created:      15 February 2005
%
%-----------------------------------------------------------------------------


\documentclass[preprint,10pt,numbers]{sigplanconf}

% The following \documentclass options may be useful:

% preprint      Remove this option only once the paper is in final form.
% 10pt          To set in 10-point type instead of 9-point.
% 11pt          To set in 11-point type instead of 9-point.
% numbers       To obtain numeric citation style instead of author/year.

\usepackage{amsmath}

\newcommand{\cL}{{\cal L}}

\usepackage{caption}
\captionsetup[figure]{name=Listing}

\usepackage{listings}
\lstdefinestyle{asm}{
  language=[x86masm]Assembler,
  basicstyle=\ttfamily,
}

\usepackage{hyperref}

\usepackage[bw]{agda}
\newcommand{\ignore}[1]{}
\providecommand{\C}{\AgdaInductiveConstructor}
\renewcommand{\C}{\AgdaInductiveConstructor}
\newcommand{\F}{\AgdaFunction}
\newcommand{\V}{\AgdaBound}
\newcommand{\labeledfigure}[3]{%
\begin{figure}%
\begin{center}%
{#3}
\caption{{#2}}
\label{#1}
\end{center}%
\end{figure}%
}

%%%%%%%%%%%%%%%%%%%%%%%%%%%%%%%%%
%                               %
% REMOVE THIS AFTER TRANSLATION %
%                               %
\usepackage{fontspec}
\setmainfont{Liberation Serif}
\setsansfont{Liberation Sans}
\newfontfamily{\cyrillicfonttt}{Liberation Mono}

\usepackage{polyglossia}
%\setmainlanguage{russian} \setotherlanguage{english}
\setmainlanguage{english} \setotherlanguage{russian}

\usepackage{etoolbox}
\newtoggle{russian-draft}

\usepackage{iflang}
\IfLanguageName{russian}{
\toggletrue{russian-draft}
}{
\togglefalse{russian-draft}
}
%                               %
%%%%%%%%%%%%%%%%%%%%%%%%%%%%%%%%%

\begin{document}

\special{papersize=8.5in,11in}
\setlength{\pdfpageheight}{\paperheight}
\setlength{\pdfpagewidth}{\paperwidth}

\conferenceinfo{CONF 'yy}{Month d--d, 20yy, City, ST, Country}
\copyrightyear{20yy}
\copyrightdata{978-1-nnnn-nnnn-n/yy/mm}
\copyrightdoi{nnnnnnn.nnnnnnn}

% Uncomment the publication rights you want to use.
%\publicationrights{transferred}
%\publicationrights{licensed}     % this is the default
%\publicationrights{author-pays}

\titlebanner{banner above paper title}        % These are ignored unless
\preprintfooter{short description of paper}   % 'preprint' option specified.
\iftoggle{russian-draft}{
\title{Формализация модели простой динамической линковки}
}{
\title{Simple dynamic linking process formalization}
}
\subtitle{Subtitle Text, if any}

\authorinfo{Name1}
           {Affiliation1}
           {Email1}
\authorinfo{Name2\and Name3}
           {Affiliation2/3}
           {Email2/3}

\maketitle

\begin{abstract}
\input{Abstract.latex}
\end{abstract}

\category{CR-number}{subcategory}{third-level}

% general terms are not compulsory anymore,
% you may leave them out
\terms
term1, term2

\keywords
keyword1, keyword2

$body$

% We recommend abbrvnat bibliography style.

\bibliographystyle{abbrvnat}

% The bibliography should be embedded for final submission.

\bibliography{bib}

\end{document}
